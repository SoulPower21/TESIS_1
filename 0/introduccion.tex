%La línea de abajo es para quitar encabezado
%\thispagestyle{plain}

\chapter*{Introducción}
\markboth{Introducción}{Introducción}
\addcontentsline{toc}{chapter}{Introducción}
% texto 
El principal objetivo de la presente investigación es desarrollar y proponer un modelo basado en inteligencia artificial para el prediagnóstico rápido y preciso del bruxismo en niños. Esto se logrará mediante la clasificación de imágenes de lengua con marcas características asociadas a esta condición. Para ello, se utilizarán algoritmos avanzados del campo del aprendizaje profundo. En particular, se implementarán Redes Neuronales Convolucionales (CNNs) y Vision Transformers, herramientas reconocidas por su capacidad de identificar patrones en imágenes y que serán clave para este proyecto.

Las imágenes de lengua que se emplearán para el entrenamiento y validación de los modelos serán extraídas de bases de datos en línea debidamente verificadas, asegurando su validez científica. El proceso de desarrollo incluirá una etapa inicial de preprocesamiento y selección de imágenes, con el objetivo de crear un conjunto de datos óptimo para el modelado. Posteriormente, se implementarán distintos algoritmos y técnicas dentro del ámbito del Deep Learning, evaluándose su desempeño mediante métricas específicas. Esto permitirá seleccionar el modelo más eficiente y eficaz, como se detalla en la metodología de la implementación.

La herramienta de inteligencia artificial que se desarrollará no busca reemplazar el diagnóstico clínico de un profesional especializado, sino convertirse en un recurso de apoyo que facilite la detección temprana de bruxismo. Esto es crucial, ya que un diagnóstico adecuado en etapas iniciales puede prevenir complicaciones graves como desgaste dental severo, dolores musculares, e incluso problemas articulares o de desarrollo facial.

El problema central que se aborda en esta investigación radica en la falta de herramientas tecnológicas específicas para el prediagnóstico de bruxismo en niños, especialmente en contextos con acceso limitado a especialistas. Además, el proceso de detección suele depender exclusivamente de la experiencia del profesional, lo que puede aumentar el margen de error o prolongar el diagnóstico. Este proyecto busca cerrar esta brecha mediante la aplicación de técnicas avanzadas de inteligencia artificial, aprovechando los recientes avances tecnológicos para ofrecer una solución eficiente y accesible.

La creciente preocupación por los problemas de bruxismo en niños, tanto en el Perú como a nivel mundial, impulsa esta investigación. Se espera que los resultados permitan mejorar los tiempos y la precisión en el diagnóstico, facilitando tratamientos oportunos y reduciendo complicaciones futuras en la salud bucodental infantil.