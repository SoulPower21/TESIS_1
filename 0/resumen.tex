%La línea de abajo es para quitar encabezado
%\thispagestyle{plain}

\chapter*{Resumen}
\markboth{Resumen}{Resumen}
\addcontentsline{toc}{chapter}{Resumen}

%RESUMEN
Con el incremento de casos de bruxismo infantil en los últimos años, especialistas en odontología e investigadores han identificado la necesidad de desarrollar métodos confiables y herramientas eficaces que permitan a los profesionales de la salud detectar y diagnosticar esta condición en etapas tempranas. Esto es especialmente relevante en contextos de clínicas u hospitales con recursos limitados y con acceso reducido a personal especializado. Un diagnóstico tardío o impreciso puede derivar en complicaciones mayores, como desgaste dental severo, problemas musculares y articulares, e incluso trastornos en el desarrollo facial.

Por ello, y considerando los avances actuales en el campo de la Inteligencia Artificial junto con el creciente acceso a herramientas computacionales de alto rendimiento, esta investigación propone el desarrollo de una aplicación móvil basada en técnicas de aprendizaje profundo para el prediagnóstico de bruxismo en niños. La herramienta analizará imágenes de la lengua con marcas características asociadas a esta condición, mejorando la precisión y rapidez del diagnóstico al mismo tiempo que facilita su implementación en entornos con recursos limitados.


\textbf{Palabras claves: } Aprendizaje profundo, Redes Neuronales Convolucionales, Visión por Computadora, imágenes, bruxismo, niños.

\clearpage
%\vspace{0.5cm}
\chapter*{Abstract}
\markboth{Abstract}{Abstract}

%LOMISMO PERO EN INGLES
With the increasing prevalence of bruxism in children in recent years, dental specialists and researchers have recognized the need to develop reliable methods and effective tools to enable healthcare professionals to detect and diagnose this condition at early stages. This is particularly important in clinics or hospitals with limited resources and reduced access to specialized personnel. Late or inaccurate diagnoses can lead to severe complications, such as advanced dental wear, muscular and joint problems, and even facial development disorders.

In light of this, and considering the current advancements in Artificial Intelligence along with the growing accessibility to high-performance computational tools, this research proposes the development of a mobile application based on deep learning techniques for the prediagnosis of bruxism in children. The tool will analyze tongue images with specific marks associated with this condition, improving the accuracy and speed of diagnosis while facilitating its implementation in resource-limited environments.


\textbf{Keywords: }Deep Learning, Convolutional Neural Networks, Computer Vision, tongue images, bruxism, children