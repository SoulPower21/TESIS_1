\subsection{Bruxismo en niños}

\subsubsection{Definición}

El término "bruxismo" proviene del griego "brygmós", que significa "apretar o rechinar los dientes". Este trastorno, también conocido como la "enfermedad silenciosa", afecta aproximadamente al 70\% de los pacientes, y su aparición se vincula principalmente al estrés y problemas psicológicos \parencite{Rojas2022}. Según lo descrito por \textcite{Rojas2022}, el bruxismo representa un desafío común en la salud dental a nivel mundial, generando alteraciones en el aparato masticatorio y contribuyendo a cambios morfológicos que comprometen la salud bucal.

Es importante destacar que el término "bruxismo" suele confundirse con otros relacionados, como bruxismo céntrico, excéntrico, nocturno, diurno, bruxomanía, parafunción, apretamiento dentario, rechinamiento dentario, entre otros \parencite{Zambra2003}.

\subsubsection{Síntomas}

Los síntomas del bruxismo en niños incluyen sonidos audibles durante el acto de apretar o rechinar los dientes, ya sea consciente o inconscientemente. Estos pueden manifestarse como chasquidos o movimientos dentales en posición céntrica o excéntrica, durante actividades como la deglución o masticación, tanto de día como de noche \parencite{vallejo2002bruxismoinfantil}. 

La magnitud de los síntomas depende de la resistencia de las estructuras involucradas y de factores como la duración, frecuencia e intensidad del bruxismo. Aunque algunas estructuras del sistema masticatorio absorben estas fuerzas sin consecuencias, otras experimentan daños que van desde el desgaste dental hasta alteraciones en los músculos masticatorios y las articulaciones temporomandibulares \parencite{vallejo2002bruxismoinfantil}.

\subsubsection{Factores de riesgo}

\paragraph{Factores odontológicos}  
El bruxismo puede estar influido por condiciones como maloclusiones esqueléticas, alteraciones oclusales y restauraciones dentales defectuosas. Sin embargo, los estudios en este ámbito han arrojado resultados inconsistentes \parencite{Zambra2003}.

\paragraph{Factores psicológicos}  
El estrés y la ansiedad son factores psicológicos clave asociados al bruxismo, especialmente en adolescentes. Un estudio indicó que los niveles elevados de catecolaminas en la orina de niños de 6 a 8 años podrían estar relacionados con altos niveles de ansiedad \parencite{vanderas1999catecholaminebruxism}. Sin embargo, otros factores como el nivel socioeconómico y cultural no muestran resultados concluyentes \parencite{serranegra2009psychosocialbruxism}.

\paragraph{Factores relacionados con el sueño}  
El bruxismo nocturno, considerado una parasomnia, ocurre durante el sueño y está relacionado con despertares parciales y altos niveles de estrés emocional. Aunque estas alteraciones son comunes en la infancia, su frecuencia persistente puede indicar problemas psicológicos subyacentes \parencite{vieiraandrade2014sleepbruxism}.

\paragraph{Factores genéticos}  
Evidencia sugiere que el bruxismo puede tener una base genética, ya que los niños cuyos padres presentaron bruxismo tienen mayor probabilidad de desarrollarlo, lo que señala una predisposición hereditaria \parencite{hublin2001parasomniasgenetics}.

\subsection{Diagnóstico y prediagnóstico del bruxismo}

\subsubsection{Método instrumental}  
La polisomnografía es considerada la técnica de referencia para diagnosticar el bruxismo durante el sueño, ya que permite un monitoreo completo de la actividad cerebral, muscular y respiratoria. Asimismo, los registros electromiográficos (EMG) y las aplicaciones móviles emergen como herramientas innovadoras para evaluar la actividad muscular en tiempo real \parencite{gutierrez2021bruxismorino}.

\subsubsection{Método no instrumental}  
El diagnóstico mediante autorreporte y evaluación clínica es ampliamente utilizado debido a su accesibilidad y bajo costo. Sin embargo, estos métodos presentan limitaciones en comparación con los instrumentales, por lo que su precisión aún debe ser mejorada \parencite{gutierrez2021bruxismorino}.

\subsection{Aprendizaje profundo y visión por computadora}

\subsubsection{Definición}  
El aprendizaje profundo, según \textcite{kim2017matlabdeeplearning}, se basa en redes neuronales con múltiples capas ocultas. Estas estructuras permiten procesar datos complejos de manera jerárquica, lo que constituye la esencia del Deep Learning.

\subsubsection{Visión por computadora}  
La visión por computadora busca replicar la capacidad humana de interpretar información visual a través de algoritmos y modelos matemáticos. Este campo, iniciado por David Marr en la década de 1980, ha evolucionado hacia aplicaciones que abarcan desde la robótica hasta el diagnóstico médico \parencite{szeliski2010computervision}.

\subsubsection{Aplicación en la salud}  
En medicina, la visión por computadora permite detectar patrones y anomalías en imágenes médicas como radiografías y resonancias magnéticas. Por ejemplo, las redes neuronales convolucionales han demostrado ser eficaces en la identificación de tumores, mejorando la velocidad y precisión del diagnóstico \parencite{litjens2017deeplearningmedical}.

\subsection{Aplicaciones móviles}

\subsubsection{Definición}  
Las aplicaciones móviles son programas diseñados para dispositivos como smartphones y tablets, ofreciendo una amplia gama de funcionalidades que van desde el entretenimiento hasta la salud \parencite{marinescu2019mobileappdev}.

\subsubsection{Importancia en la salud}  
Estas aplicaciones han revolucionado la gestión de la salud al ofrecer herramientas personalizadas para el monitoreo, la consulta y la toma de decisiones informadas. Además, promueven la autonomía de los pacientes al proporcionar acceso inmediato a información confiable \parencite{marinescu2019mobileappdev}.
