\subsection{Hipótesis General}
HG: \newcommand{\HipotesisGeneral}{
	El uso de un sistema automatizado basado en técnicas de aprendizaje profundo, como las redes neuronales convolucionales (CNN), permite realizar un prediagnóstico efectivo del bruxismo en niños mediante la identificación de patrones visuales en imágenes de la lengua, con una precisión comparable o superior a los métodos diagnósticos tradicionales.
	}
\HipotesisGeneral


\subsection{Hipótesis Específicas}
\newcommand{\Hone}{
	Un modelo de aprendizaje profundo, específicamente una red neuronal convolucional (CNN), puede clasificar imágenes de la lengua con alta precisión para detectar signos de bruxismo en niños.
	}
\newcommand{\Htwo}{
	Existen patrones visuales distintivos en las imágenes de la lengua que están relacionados con episodios de bruxismo en niños.
	}
\newcommand{\Hthree}{
	Las técnicas de procesamiento de imágenes, como la segmentación y el análisis de texturas, son efectivas para extraer características relevantes de las imágenes de la lengua para el diagnóstico de bruxismo.
	}
\newcommand{\Hfour}{
	El sistema automatizado basado en aprendizaje profundo logra una precisión, sensibilidad y especificidad competitivas en la detección de bruxismo en niños en comparación con los métodos tradicionales.
	}
\newcommand{\Hfive}{
	La implementación de un sistema automatizado para el análisis de imágenes de la lengua incrementa la tasa de detección temprana de bruxismo en niños, facilitando su diagnóstico en entornos clínicos y domésticos.
		}

\begin{itemize}
	\item HE1: {\Hone}
	\item HE2: {\Htwo}
	\item HE3: {\Hthree}
	\item HE4: {\Hfour}
	\item HE5: {\Hfive}
\end{itemize}