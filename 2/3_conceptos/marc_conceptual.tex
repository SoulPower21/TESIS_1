\subsection{Bruxismo primario}
El bruxismo primario, también conocido como idiopático, es un trastorno caracterizado por el hábito inconsciente de apretar o rechinar los dientes, especialmente durante el sueño. A diferencia del bruxismo secundario, no se asocia a una causa médica o psicológica identificable. Se cree que factores genéticos, estrés y trastornos del sueño podrían estar involucrados en su desarrollo. A largo plazo, el bruxismo primario puede causar desgaste dental, dolor en la mandíbula y otros problemas bucales \parencite{kato2001epidemiologybruxism}.

\subsection{Disfunción temporomandibular}
La disfunción temporomandibular (DTM) se caracteriza por diversos síntomas, incluyendo dolor en la región bucal, sonidos en las articulaciones y restricciones en los movimientos de la mandíbula. Esta condición tiene múltiples causas, que abarcan desde traumatismos y hábitos parafuncionales hasta el estrés, todos los cuales pueden exceder la capacidad fisiológica de los individuos. Factores como la estabilidad articular, la oclusión dental y características genéticas también son determinantes. Un estudio indica que hay una relación entre el bruxismo durante el sueño y la presencia de dolor en pacientes con DTM, lo que enfatiza la necesidad de una evaluación exhaustiva para su tratamiento \parencite{blanco2014selfreportbruxism}.

\subsection{Cefaleas}
Las cefaleas son uno de los trastornos más frecuentes del sistema nervioso, afectando a aproximadamente la mitad de los adultos en el último año y siendo la sexta causa de incapacidad en el mundo. Aunque la mayoría no son graves, es importante buscar atención médica si el dolor es intenso y repentino, o si se acompaña de síntomas neurológicos o fiebre \parencite{clinica2018cefalea}.

\subsection{Sensibilidad dental}
La hipersensibilidad dentinaria (HD), también conocida como sensibilidad dental, se caracteriza por un dolor intenso y temporal en los dientes, que ocurre debido a la exposición de la dentina al entorno bucal. Este tipo de dolor se desencadena por estímulos externos como alimentos o bebidas frías, calientes, ácidas o dulces, así como por la presión táctil \parencite{dentaid2024sensibilidad}.

\subsection{Odontología del sueño}
La medicina dental del sueño es una especialidad odontológica que se ocupa del tratamiento de los trastornos del sueño, siendo el síndrome de apnea obstructiva del sueño (SAOS) y los ronquidos las afecciones más comunes tratadas \parencite{gutierrez2021bruxismorino}.

\subsection{Lingua geográfica}
La lengua geográfica es una condición inflamatoria que, aunque no representa un riesgo, se manifiesta en la superficie de la lengua. En esta afección, la lengua presenta pequeñas protuberancias de color blanco rosado, conocidas como papilas, que son estructuras finas similares a vellos. En las áreas afectadas por la lengua geográfica, las papilas están ausentes, dejando zonas suaves y rojas, a menudo con bordes ligeramente elevados \parencite{mayoclinic2024lenguageografica}.

\subsection{Redes neuronales convolucionales}
\subsubsection{Definición y funcionamiento}
Las redes neuronales convolucionales (CNN) son un tipo de red neuronal organizadas en varias capas, conocidas como multicapa. Estas redes presentan una arquitectura donde las neuronas están interconectadas. 

\subsubsection{Componentes principales}
Se componen de una capa de convolución, donde cada neurona filtra la información de entrada para generar un mapa de características; seguida por una capa de submuestreo, que reduce el tamaño de la información procesada; y finalmente, una capa completamente conectada que se encarga de la clasificación. Las CNN son especialmente eficaces en proyectos de procesamiento de imágenes gracias a su capacidad de clasificación \parencite{lopezpacheco2021redesconv}.

\subsection{Support Vector Machines (SVM)}
Las Support Vector Machines son un conjunto de métodos de aprendizaje supervisado utilizados para clasificación y regresión. Se basan en encontrar un hiperplano óptimo que separa diferentes clases en un espacio multidimensional. Este hiperplano maximiza el margen entre las clases más cercanas, llamadas vectores de soporte. Las SVM son efectivas en espacios de alta dimensión y pueden utilizarse con distintos tipos de núcleos (kernels) para manejar problemas no lineales \parencite{cortes1995svm}.

\subsection{Extracción de características}
La extracción de características es el proceso de transformar datos en bruto y complejos en representaciones numéricas más simples y significativas, llamadas características. Estas características son luego utilizadas como entrada para algoritmos de aprendizaje automático, permitiendo a las máquinas "entender" y aprender de los datos \parencite{szeliski2010computervision}.

\subsection{Transfer learning}
El \textit{transfer learning} permite utilizar modelos preentrenados en un nuevo conjunto de datos, facilitando la mejora del rendimiento cuando se dispone de pocos datos. Esto ahorra tiempo y recursos durante el entrenamiento, aprovechando el conocimiento previo \parencite{pan2010survey}.

\subsection{Data augmentation}
El \textit{data augmentation} amplía un conjunto de datos mediante transformaciones como rotaciones y escalados. Esta técnica mejora la robustez del modelo y reduce el riesgo de sobreajuste, creando un conjunto más diverso para el entrenamiento \parencite{shorten2019survey}.

\subsection{Segmentación de imágenes}
La segmentación de imágenes divide una imagen en regiones significativas para facilitar la identificación de objetos. Es esencial en tareas de detección y reconocimiento, utilizando redes neuronales para lograr alta precisión \parencite{litjens2017deeplearningmedical}.

\subsection{Redes generativas adversarias}
Las redes generativas adversarias (GANs) consisten en un generador y un discriminador que compiten entre sí. El generador crea datos sintéticos, mientras que el discriminador evalúa su autenticidad. Son efectivas en la generación de imágenes y la mejora de resolución \parencite{goodfellow2014generative}.

\subsection{Pooling}
El \textit{pooling} es una técnica utilizada en redes neuronales convolucionales (CNN) para reducir la dimensionalidad de los mapas de características generados por las capas de convolución. Este proceso consiste en aplicar una función de agregación, como la máxima (\textit{max pooling}) o la media (\textit{average pooling}), sobre bloques de características. Ayuda a disminuir la carga computacional, mejora la generalización del modelo y hace que las redes sean menos sensibles a pequeñas variaciones en la entrada \parencite{boureau2010theoretical}.

\subsection{Regularización}
La regularización previene el sobreajuste en modelos de aprendizaje automático. Incluye técnicas como la penalización de coeficientes y el \textit{dropout}, que apaga neuronas aleatoriamente durante el entrenamiento para mejorar la generalización del modelo \parencite{srivastava2014dropout}.
